\AtBeginDocument{%
\begingroup\pagestyle{empty}\raggedleft\parindent0pt

\vspace*{5cm}

\huge{\textbf{A noite não é eterna}} \hspace*{1.25cm}

\clearpage\endgroup

%% Créditos ------------------------------------------------------

\begingroup
\pagestyle{empty}
\begin{Parskip}
\textbf{COLEÇÃO TRÁS-OS-MARES}

\bigskip

\textbf{coordenação}

Renato Rezende e Maria João Cantinho

\bigskip

\textbf{projeto gráfico}

Sergio Cohn

\bigskip

\textbf{capa}

Lucio Ayala

\bigskip

\textbf{revisão}

Paulo Henrique Pompermaier

\bigskip

\textbf{distribuição}

Editora Hedra

\bigskip

\textbf{edição adotada}

\emph{A noite não é eterna}, Alfragide, Oficina do livro, 2016

\bigskip

\textbf{\textbf{Com o apoio da Direção-Geral do Livro, dos Arquivos e das \mbox{Bibliotecas} -- \textsc{dglab} / Cultura / Portugal}}

\bigskip

\textbf{Dados internacionais de Catalogação na Publicação -- CIP}

S586\\
Silva, Ana Cristina\\
A noite não é eterna / Ana Cristina Silva. -- Rio de Janeiro: Circuito; Lisboa: \versal{DGLAB}, 2020. (Coleção Trás-os-mares).\\
130 p.\\[2pt]

\versal{ISBN} 978-65-8697-404-1\\[2pt]

1. Literatura Portuguesa. 2. Romance. I. Título. II. Série. III. Direção-Geral do Livro, dos Arquivos e das Bibliotecas (\versal{DGLAB}).\\
\versal{CDU} 821.134.3 \quad \versal{CDD} 869.3

\vspace*{\fill}

2020

www.editoracircuito.com.br
\end{Parskip}

\pagebreak
%% Front ---------------------------------------------------------
% Titulo
\begin{flushright}
 
\vspace*{5cm}

\huge{\textbf{A noite não é eterna}} \hspace*{1.25cm}

\LARGE{\versal{Ana Cristina Silva}} \hspace*{2.82cm}

%{{\footnotesize{} \ifdef{\numeroedicao}{\numeroedicao}{1}ª edição} \par}

%logos
\vfill
\hfill\includegraphics[width=5cm]{logos.pdf}\\ \normalsize{2020}
%\includegraphics[width=.4\textwidth,trim=0 0 25 0]{logo.jpg}\\\smallskip
\par\end{flushright}\clearpage
% Resumo -------------------------------------------------------
\begingroup \footnotesize \parindent0pt \parskip 5pt \thispagestyle{empty} \vspace*{.25\textheight}\mbox{} \vfill
\baselineskip=.92\baselineskip
\IfFileExists{PRETAS.tex}{\input{PRETAS}}{% 
\ifdef{\resumo}{\resumo\par}{}
\ifdef{\sobreobra}{\sobreobra}{}
\ifdef{\sobreautor}{\mbox{}\vspace{4pt}\newline\sobreautor}{}
\ifdef{\sobretradutor}{\newline\sobretradutor}{\relax}
\ifdef{\sobreorganizador}{\vspace{4pt}\newline\sobreorganizador}{\relax}\par}
\thispagestyle{empty} \endgroup
\ifdefvoid{\sobreautor}{}{\pagebreak\ifodd\thepage\paginabranca\fi}
% Sumário -------------------------------------------------------

\sumario{}
\IfFileExists{INTRO.tex}{\include{INTRO}}

\IfFileExists{TEXTO.tex}{\mbox{}\include{TEXTO}}\endgroup
%\part[{{\def\break{}\titulo}}]{\titulo}
} % fim do AtBeginDocument

% Finais -------------------------------------------------------
\AtEndDocument{%


\pagebreak\ifodd\thepage\paginabranca\fi

\ifdef{\imagemficha}{\IfFileExists{\imagemficha}{\includegraphics[width=.7\textwidth]{\imagemficha}\par}}{}

\mbox{}\vfill\small\thispagestyle{empty}
\begin{center}
\begin{minipage}{.8\textwidth}
\centering\tiny\noindent{}Adverte-se aos curiosos que se imprimiu este livro \ifdef{\grafica}{na gráfica \grafica}{em nossas oficinas}, 
em \today \ifdef{\papelmiolo}{em papel \papelmiolo}, em tipologia \tipopadrao{}, com diversos sofwares livres, 
entre eles, Lua\LaTeX, git \& ruby. \ifdef{\RevisionInfo{}}{\par(v.\,\RevisionInfo)}{}\par \begin{center}\normalsize\adforn{64}\end{center}
\end{minipage}
\end{center}
}
